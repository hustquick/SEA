\documentclass[12pt]{letter}
\usepackage{xcolor}
\usepackage{framed}
\definecolor{shadecolor}{RGB}{180,180,180}

\begin{document}

\emph{Reviewers comments:}

Reviewer \#1:  This manuscript can be accepted for publication after minor revisions, see the followings:

\begin{enumerate}
\item English should be improved.

\begin{snugshade*}
\emph{Answer to the reviewer:} The manuscript has been carefully checked and the errors found have been corrected.
\end{snugshade*}

\item The Abstract should be improved.

\begin{snugshade*}
\emph{Answer to the reviewer:} The abstract has been improved with revised Stirling engine model and clearer conclusion. E.g., 'flow order has little influence on the SEA performance' has been mentioned in the abstract as part of the conclusion.
\end{snugshade*}

\item Better description and explanation on the figures.

\begin{snugshade*}
\emph{Answer to the reviewer:} Description and explanation of the figures have been modified so that the figures can be more easily understood. E.g., for Figure 4, the sentence 'Figure 4 shows the schematic of a Stirling cycle, process 1-2 and process 3-4 are the two isothermal processes, process 2-3 and process 4-1 are the two isochoric processes.' has been added to describe the figure clearly.
\end{snugshade*}

\item Better description and explanation on the Tables.

\begin{snugshade*}
\emph{Answer to the reviewer:} Description and explanation of the tables have been modified so that the tables can be more easily understood. E.g., for Table 4, the reasons of chosen cooling fluid type and the value of $s_{se}$ and $p_{se}$ were given in the manuscript.
\end{snugshade*}

\item The Conclusion could, also, be improved.

\begin{snugshade*}
\emph{Answer to the reviewer:} The conclusion part has been rewritten. The idea of using SEA to overcome the limitation of small engine capacity in dish-Stirling engine system was restated in the conclusion part. The conclusions from the result analysis part was briefly introduced. A further research outlook was added in the last paragraph.
\end{snugshade*}

\item Introduction part needs to be extended by some of the recent published papers to show the importance of thermodynamic analysis for Stirling engine in high quality journals. The following references should be included in this manuscript:

[1] Ahmadi, Mohammad H., Hoseyn Sayyaadi, Amir H. Mohammadi, and Marco A. Barranco-Jimenez. "Thermo-economic multi-objective optimization of solar dish-Stirling engine by implementing evolutionary algorithm." Energy Conversion and Management 73 (2013): 370-380. 

[2] Ahmadi, Mohammad Hossein, Hoseyn Sayyaadi, Saeed Dehghani, and Hadi Hosseinzade. "Designing a solar powered Stirling heat engine based on multiple criteria: maximized thermal efficiency and power." Energy Conversion and Management 75 (2013): 282-291.

[3] Patel, Vivek, and Vimal Savsani. "Multi-objective optimization of a Stirling heat engine using TS-TLBO (tutorial training and self learning inspired teaching-learning based optimization) algorithm." Energy 95 (2016): 528-541.

[4] Luo, Zhongyang, Umair Sultan, Mingjiang Ni, Hao Peng, Bingwei Shi, and Gang Xiao. "Multi-objective optimization for GPU3 Stirling engine by combining multi-objective algorithms." Renewable Energy 94 (2016): 114-125.

[5] Ahmadi, Mohammad Hossein, Mohammad Ali Ahmadi, Adel Mellit, Fathollah Pourfayaz, and Michel Feidt. "Thermodynamic analysis and multi objective optimization of performance of solar dish Stirling engine by the centrality of entransy and entropy generation." International Journal of Electrical Power \& Energy Systems 78 (2016): 88-95.

[6] Li, Ruijie, Lavinia Grosu, and Diogo Queiros-Conde. "Multi-objective optimization of Stirling engine using Finite Physical Dimensions Thermodynamics (FPDT) method." Energy Conversion and Management 124 (2016): 517-527.

\begin{snugshade*}
\emph{Answer to the reviewer:} The introduction part has been rewritten in a more logical way. Studies on Stirling engine analysis have been classified to show the importance of thermodynamic analysis for Stirling engine. The suggested references have been added as a new paragraph to show one of the recent research focused area of using multi-objective optimization algorithms to obtain a better performance for Stirling engine.
\end{snugshade*}

\item I hope that the authors refer to more published papers in the APEN.

\begin{snugshade*}
\emph{Answer to the reviewer:} Thank you for your suggestion. More related published papers in the APEN have been referred in the paper.
\end{snugshade*}

\item Format of the reference list should be checked and prepare according to the APEN rules.

\begin{snugshade*}
\emph{Answer to the reviewer:} The bibliography style of the reference has been changed from 'unsrt' style to the required 'model3-num-names.bst' style.
\end{snugshade*}

\end{enumerate}

\newpage
Reviewer \#2:  The paper attempts to highlight the effectiveness of the so-called Stirling engine array (SEA) in the solar dish power plants. A simplified thermodynamic model justifying the system performance is presented to study the system behavior taking into account different connection types of the solar dishes. At the end of the day, a simulation study is carried out to investigate the best candidate connection method among the considered configurations. Although, the paper has some merits to be considered as a journal paper it is not enough for such an eminent international Journal. I can not recommend the paper for publication because of the following reason:

\begin{enumerate}
\item There are several grammatical and English problems throughout the paper. As an example please see line 281, "this is an reasonable…" should be "this is a reasonable…" and so on.

\begin{snugshade*}
\emph{Answer to the reviewer:} 
After carefully checking the manuscript, we revised the errors found.
\end{snugshade*}

\item It is conventional to have the solar dish equipped with a compact multi-cylinder free piston Stirling engine (FPSE) at the focal point. In order to achieve a higher output power, an array of such modules (a solar dish + an FPSE) is conventionally considered. However, what the paper is taking about is a different methodology. Thus, a purely analytical result is not sufficient to prove its performance. It was expected to see some experimental data to support the analytical results.

\begin{snugshade*}
\emph{Answer to the reviewer:} 
This paper presents a new way to overcome the limitation of small capacity engine of the conventional dish-Stirling system. The Stirling engines can reach higher capacities when they are put on the ground. Due to the limitation of present experiment condition, the proposed ideas were proved by analytical results. To increase the reliability, the proposed Stirling engine model was validated by experiment data and compared with previous models. In the future work, some experiments will be required to support the analytical results.
\end{snugshade*}

\item The presented mathematical model dose not take into account the Stirling engine type (e.g. Alfa, Betta, Gamma, FPSE etc). The convective heat transfer between the working gas (in the engine) and the heat exchanges has not been considered in the presented model. The most important parameter affecting the engine power and efficiency is the working gas temperature in the engine cylinder; not just the liquid temperature in the inlet and outlet of the array.

\begin{snugshade*}
\emph{Answer to the reviewer:} 
The Stirling engine model was rebuilt with a $\beta$ type engine. The convective heat transfer between the working gas (in the engine) and the heat exchangers has been considered in the new model. The working gas temperature in the engine cylinder was considered as an important parameter in the new model.
\end{snugshade*}

\item In the simulation results there are not enough justifications for the trends of the curves. For instance, in Fig.6 why Types 1 and 5 show initially different trends than the rest of the configurations? Why some discontinuities are seen?

\begin{snugshade*}
\emph{Answer to the reviewer:} 
There are explanation for the discontinuities in Fig. 6. Please see in line 474 to 480 (line 340 to 348 of the previous version paper), 'For some types of SEA, when $T_{i,h}$ is lower than a critical temperature, some of the engines in the SEA will not work and there will be turning points on the $\eta-T_{i,h}$, $P-T_{i,h}$ curves. E.g. for SEA of Type 1, when $T_{i,h}$ is lower 820\,K, all the engines stop working, turning points at 820\,K can be found on the $\eta-T_{i,h}$, $P-T_{i,h}$ curves in Figure 6.'
\end{snugshade*}

\item The trends of the curves in all graphs do not imply a new finding. It is clear that the power and efficiency of a Stirling engine is increased as the hot source temperature is increased and the sink temperature is reduced.

\begin{snugshade*}
\emph{Answer to the reviewer:} 
The aim of the curves is not trying to find out the relationship between the performance of a specified connection type of SEA and the fluid temperatures. It is trying to find out the performance differences of different connection type of SEAs.
\end{snugshade*}

\item However, the obtained results regarding the performance of each connection types were interesting. The authors are encouraged to provide an experimental rig to support the analytical data and resubmit the paper. 

\begin{snugshade*}
\emph{Answer to the reviewer:} 
Due to the limitation of present experiment condition, the proposed ideas are proved by analytical results. To increase the reliability, the proposed Stirling engine model was validated by experiment data and compared with previous models. As it is mentioned in the paper, in the future researches, the experiments of influence of connection type on SEA’s performance can be carried out to verify the conclusions in this paper.
\end{snugshade*}

\end{enumerate}

\newpage
Reviewer \#3:  This paper discussed new arrangements of SEA as described by a combination of types of connection of hot and cold stream of fluids. The analytical models for Stirling cycle and Stirling engine were developed. A set of parameters was selected and the modeling results are discussed such that the best performance for a SEA arrangement can be selected.

The reviewer found that the attempt to come up with a new arrangement of SEA to counter the suggested limitation of dish-Stirling system is interesting. 

\begin{enumerate}

\item The analytical model for Stirling cycle and Stirling engine - from thermodynamics viewpoint - are, however, too simplistic. With the goal of the study in mind, the reviewer found insufficient support for the reliability of the chosen models for such a goal. In addition, considering the nature of dish-Stirling system, the lack or insufficiency of  fluid dynamics consideration is one of the major concern for this study. 

\begin{snugshade*}
\emph{Answer to the reviewer:} 
Admittedly, the analytical models in the previous paper were too simple, and the mentioned deficiencies of the models needed to be concerned.

In the new paper, new analytical Stirling engine model was developed with consideration of thermodynamics of the working gas in the engine. Imperfect heat transfer, effects of finite finite speed of piston, mechanical friction, pressure drop, internal conduction and shuttle conduction were considered in the new model. The model was validated using prototype GPU-3 Stirling engine experimental data.
\end{snugshade*}

\item In the modeling section, there is no support for the choice of parameters used in this study (Table 1). For the Result Analysis section, there is no validation of the results against prior simulation results or experimental results in any way. 

\begin{snugshade*}
\emph{Answer to the reviewer:} 
Explanation of the choice of parameters used in the modeling section was added in the new paper. To eliminate interference of other factors, heating and cooling fluids are chosen to have same parameters for different connection types of SEAs. To clearly find out the performance differences of different SEAs, large temperature differences of the heating/cooling fluids after heat exchange with the engines are preferred. Air was chosen as the cooling fluid instead of commonly used water to avoid small temperature rise and evaporation in the cooling process. Rotation speed of the engines and mean effective pressure were chosen to be $25\,\mathrm{Hz}$ and $5\,\mathrm{MPa}$ respectively to get the best Stirling engine model for performance prediction, as pointed in section 3.2.

Validation of the proposed Stirling engine model was carried out with the experiment data of prototype GPU-3 Stirling engine. Performance prediction of the proposed model was compared with two classic models, and results shown that the proposed model had better agreement with the experiment results.
\end{snugshade*}

\end{enumerate}

\newpage
Reviewer \#4:  This article reports on a theoretical activity on the integration of multiple Stirling engine for power generation, with possible application in solar plants. The argument would be of interest of Applied Energy, because it deals with a topic covered by the journal. However, the contribution is very little, so the article does not reach the minimum threshold for a research paper to be published in an important journal.

\begin{enumerate}

\item The Introduction reviews the mathematical models for predicting the performance of a Stirling engine (more than 20 citations), but the review is quite qualitative and not finalized to the aim of the research, i.e. coupling more Stirling engine (SE) for improving the power and the efficiency of the solar plant. Furthermore the reason why multiple SE should be connected in a network is not explained.

\begin{snugshade*}
\emph{Answer to the reviewer:} 
The introduction part has been revised to be more logical. First, the paper introduces the dish-Stirling system and shows its drawback. Second, a novel idea of using SEA to overcome the drawback has been proposed. Third, the researches of using dish collector to collect heat for other applications are introduced. Forth, The Stirling engine analysis and Stirling engine models are reviewed. The Stirling engine numerical model is introduced from Simple model and Simple II model to their alternative methods or improvements with consideration of various losses.  The researches of using finite-time thermodynamics and multi-objective optimization algorithms to get the performance of the Stirling engine are also reviewed. Finally, the way to find out the performance difference of different connection types of SEA is put forward by using existing Stirling engine modeling methods.

The reason why multiple SE should be connected in a network can be found in line 24 to 34. (line 27 to 35 of the previous version paper)
\end{snugshade*}

\item The authors show 5 possible schemes of serial/parallel connection of 2 SE. Therefore, the combination of a larger number of engines would be very high (factorial law). In contrast, the authors report only 5 curves (Fig. 9), even in presence of 15 SE. This is a unjustified shortcoming.

\begin{snugshade*}
\emph{Answer to the reviewer:} 
As mentioned in section 2, there are five basic connection types for an SEA. All other connection types are the combination of these five basic connection types. This paper investigates the five basic connection types. Please see in line 224 to 233 (line 189 to 195 of the previous version paper). 
\end{snugshade*}

\item The details about their own model of single SE are not reported. Furthermore, the model appears very limited with respect to currently available mathematical models and based only on thermodynamic predictions without accounting for irreversible phenomena (frictions, fluid-dynamics).

\begin{snugshade*}
\emph{Answer to the reviewer:} 
The Stirling engine model has been improved by considering imperfect heat transfer, effects of finite finite speed of piston, mechanical friction, pressure drop, internal conduction and shuttle conduction. Details of the model are explained in the paper.
\end{snugshade*}

\item Taking into account that the first 5 figures are qualitative and not-relevant, the added value of the article is poor and only limited to results about the possible schemes of connection for SE in Figs 6-9 with a not surprising finding that counter-current connection would be better.

\begin{snugshade*}
\emph{Answer to the reviewer:} 
Figure 1 shows the idea of using SEA to overcome the drawback of dish-Stirling system. Figure 2 shows the classified five basic connection types according to the direction-irrelevant feature of Stirling engine. These are the innovations of this paper. Figure 3 shows an instance of connection type of an SEA to prove the idea of 'All other connection types are the combination of these five basic connection types'. And this is the reason why only five basic connection types were investigated numerically. Figure 4 shows the regeneration process of the Stirling cycle. Figure 5 was replaced by flowcharts of the SEA. Figure 6-9 show the results of simulations of different connection types of SEA under different parameters. The result is not so intuitive for the following reasons: Using parallel flow, on the one hand, will reduce the flow rate of the fluid, which will reduce the power of each engine; however, on the other hand, will take the advantage of higher inlet heating fluid temperature (or lower inlet cooling fluid temperature), which may increase the power of each engine. Using serial flow, on the one hand, will increase the flow rate of the fluid, which will increase the power of each engine; however, on the other hand, the inlet heating fluid temperature reduces with the flow direction (or the inlet cooling fluid temperature increases with the flow direction), which leads to lower engine power along the flow direction. Using the same order will lead to largest fluid temperature difference (temperature difference of the heating and cooling fluids) at the first engines and smallest fluid temperature difference at the last engines. Using the reverse order will lead to more averaged fluid temperature differences of the engines. The reverse order (counterflow), which leads to a smaller fluid temperature difference, has a better heat transfer effect for its lower exergy loss. However, for a Stirling engine, the smaller fluid temperature difference leads to lower performance due to the lower temperature difference of the working gas in the hot space and cold space.
\end{snugshade*}

\item There is no experimental validation of the exposed results 

\begin{snugshade*}
\emph{Answer to the reviewer:} 
The new Stirling engine model was validated by the experiment data of GPU-3 Stirling engine and compared with previous models.
\end{snugshade*}

\item The symbols are not explained in the text (only in the Nomenclature).

\begin{snugshade*}
\emph{Answer to the reviewer:} 
Explanations of the symbols are added in the text.
\end{snugshade*}

\item The text should be revised by a mother-tongue assistant.

\begin{snugshade*}
\emph{Answer to the reviewer:} 
The text is carefully checked and revised by a mother-tongue friend.
\end{snugshade*}

\end{enumerate}

\end{document}
